% Entfernt Probleme durch Verwendung von Koma-Script Klassen (scrreport)
\usepackage{scrhack}
% Erlaubt Konfiguration von Header & Footer
\usepackage[automark,headsepline,footsepline,plainheadsepline,plainfootsepline]{scrlayer-scrpage}

% Anpassung an Sprache (deutsch)
\usepackage[ngerman]{babel}

% Umlaute
\usepackage[latin1]{inputenc}
\usepackage[T1]{fontenc}
\usepackage{textcomp} % Euro-Zeichen etc.

% Schrift
\usepackage{lmodern}
\usepackage{relsize}

% Einbinden von JPG-Grafiken erm�glichen
\usepackage[dvips,final]{graphicx}

% Erm�glichen mathematischer Symbole
\usepackage{amsmath,amsfonts}

% F�r die Definition der Zeilenabst�nde, Seitenr�nder etc.
\usepackage{setspace}
\usepackage{geometry}

% URL-Unterst�tzung
\usepackage{url}

\usepackage{float}

% Abk�rzungsverzeichnis 
% Alles weitere hierzu in: "Inhalt\Abkuerzungen.tex".
\usepackage[intoc]{nomencl}
\let\abbrev\nomenclature
\renewcommand{\nomname}{Abk�rzungsverzeichnis}
\setlength{\nomlabelwidth}{.15\textwidth}

% PDF-Optionen -----------------------------------------------------------------
\usepackage[
    bookmarks, % Es werden Bookmarks verwendet
    bookmarksopen=true, % Farbe von Bookmarks
    colorlinks=true, % Farbe von Verkn�pfungen
    linkcolor=black, % einfache interne Verkn�pfungen
    anchorcolor=black, % Ankertext
    citecolor=black, % Verweise auf Literaturverzeichniseintr�ge im Text
    filecolor=black, % Verkn�pfungen, die lokale Dateien �ffnen
    menucolor=black, % Acrobat-Men�punkte
    urlcolor=black, % Farbe der URLs
    plainpages=false, % zur korrekten Erstellung der Bookmarks
    pdfpagelabels, % zur korrekten Erstellung der Bookmarks
    hypertexnames=false, % zur korrekten Erstellung der Bookmarks
    linktocpage, % Seitenzahlen anstatt Text im Inhaltsverzeichnis verlinken
    pdfusetitle % Erm�glicht das Setzen der Meta-Daten des erzeugten PDFs
]{hyperref}
% \renewcommand{\theHsection}{\thepart.section.\thesection}
% \hypersetup{
%     %pdftitle={\titel \untertitel},
%     pdfauthor={\autor},
%     pdfcreator={\autor}
%     %pdfsubject={\titel \untertitel},
%    % pdfkeywords={\titel \untertitel}
% }

% Wird f�r Teile der Formatierung des Deckblatts und die Verwendung von
% Aufz�hlungen ben�tigt
\usepackage{listings}
\usepackage{xcolor} 
\usepackage{tabularx}


% fortlaufendes Durchnummerieren der Fu�noten
\usepackage{chngcntr}

% bei der Definition eigener Befehle ben�tigt
\usepackage{ifthen}

% sorgt daf�r, dass Leerzeichen hinter parameterlosen Makros nicht als Makroendezeichen interpretiert werden
\usepackage{xspace}

% F�r das Erstellen eines Glossars
\usepackage[toc, automake, nonumberlist]{glossaries}

\usepackage[figure,table,lstlisting]{totalcount}