\chapter{Problemstellung}
\label{old:Problemstellung}

Rundenbasierte Strategiespiele sind seit geraumer Zeit ein fundamentaler Bestandteil der Videospiel-Industrie. Sie existieren in vielen Variationen und der Erfolg des Spielers ist ma�geblich von der gew�hlten Strategie abh�ngig. Seit den 90er Jahren wurden auch vermehrt computergesteuerte Spieler als Gegner des Menschen eingesetzt. Hierf�r l�sst sich zum Beispiel die Entwicklung von KI-Spielern im Rundenbasierten Strategiespiel Civilization anf�hren \cite{Davis1999}.\\
\\
Was damals noch eine Herausforderung f�r die Entwickler dargestellt hat und Teil der aktiven Forschung war, ist heute Alltag geworden und beinahe jedes Strategiespiel, ob rundenbasiert oder kontinuierlich, verf�gt �ber computergesteuerte Spieler. Aufgrund der vermehrten Anwendung solcher KI-Spieler in der Praxis, sollte die effektive Entwicklung dieser Einzug in die Lehre erhalten. Hierbei ist es wichtig, das Wissen m�glichst Praxisnah und effizient den Studenten zu vermitteln. Hierf�r bietet es sich an, den Studenten eine M�glichkeit zur Verf�gung zu stellen, die Ihnen hilft, eigens entwickelte KIs zu testen und gegeneinander antreten zu lassen. \\
\\
Aus dieser Idee ergeben sich mehrere Anforderungen, deren Erf�llung eine solche Software gew�hrleisten und gleichzeitig die Problemstellung verdeutlichen:

\begin{enumerate}
	\item Die Software muss mindestens zwei KI-Spieler zulassen. 
	\item Die Software muss ein m�glichst faires Spiel bereitstellen.
	\item Die Software muss eine einfache Integration neuer KIs erm�glichen und einen schnellen Austausch dieser gew�hrleisten. 
	\item Die Software muss rundenbasiertes Spielen erm�glichen. 
	\item Die Software muss strenge Regeln formulieren, sodass KIs kein unerw�nschten Verhalten lernen, wobei sie fehlerhafte Regeln ausnutzen w�rden. 
	\item Die Software muss einen Sieger feststellen. 
\end{enumerate}

Aus der obigen Konkretisierung der Problemstellung l�sst sich erkennen, dass die genannte Software sowohl das Spiel an sich umfassen, als auch eine Schnittstelle f�r die entwickelten KI-Spieler bereitstellen muss. Insgesamt l�sst sich die Problemstellung also so zusammenfassen, dass eine Software entwickelt werden muss, die aus zwei Komponenten besteht (rundenbasiertes Strategiespiel + KI-Schnittstelle) und gleichzeitig die genannten Anforderungen erf�llt. 

