% Vor dem Hauptteil werden die Seiten in gro�en r�mischen Ziffern nummeriert.
\pagenumbering{roman}

\section*{Kurzfassung}
\label{sec:Kurzfassung}
Die folgende Projektarbeit befasst sich mit der Studienplatzvergabe indischer Hochschulen. Diese unterlag lange Zeit einem Problem: Es gab deutlich mehr Bewerber als Pl�tze und dennoch blieben viele dieser Pl�tze unbesetzt. Ziel ist es daher dem Leser einen Einblick in die Studienplatzvergabe Indiens zu geben und  anhand des L�sungsvorschlages von (vgl. \cite[S. 1ff]{Baswana2015}) Probleme dieser darzustellen. Zudem soll das Verfahren implementiert, sowie ein Vergleich zu verschiedenen Alternativen gezogen werden. Auf dieser Grundlage wurden drei Forschungsfragen formuliert: \\

\fett{RQ1:} Wie wird versucht, die Herausforderungen der Studienplatzvergabe in Indien zu bew�ltigen?\\
\fett{RQ2:} Wie kann das vorgestellte Verfahren implementiert werden? \\
\fett{RQ3:} Welche Alternativen Ans�tze existieren zur L�sung des Problems der Platzvergabe an Hochschulen?\\

\fett{Resultat der Arbeit:} Das gezeigte Verfahren erf�llt alle gestellten Anforderungen und ist in der Lage die frei bleibenden Pl�tze um ca. 70\% zu reduzieren. Dennoch existieren Verbesserungsm�glichkeiten, die zum Teil aus den Alternativen hervorgehen. Generell k�nnen aber auch Schl�sse in Bezug auf Quotenregelungen und allgemein Sitzplatzvergabe aus dem Verfahren gewonnen werden. Die Implementierung ist trotz der Vielzahl der Anforderungen nicht komplexer als f�r vergleichbare Algorithmen.

