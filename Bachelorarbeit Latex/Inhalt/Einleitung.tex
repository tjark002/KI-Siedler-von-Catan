\chapter{Einleitung}
\label{cha:Einleitung}
K�nstliche Intelligenz (KI) ist heutzutage in Computerspielen weit verbreitet und fand schon Ende des letztes Jahrtausends Einzug in die Spieleindustrie (\cite[vgl. S. 1]{Davis1999}). Seitdem haben sich die KIs weiterentwickelt und agieren heute besser als damals. Dennoch ist die Entwicklung von starker KI f�r Strategiespiele aufwendig, weshalb h�ufig KIs gebaut werden, die durch das \quotes{Cheaten} einen Vorteil gegen�ber menschlichen Spielern haben (vgl. \cite[vgl. S. 24]{Ruiz2007}). 

In dieser Arbeit soll das bekannte Gesellschaftsspiel \textit{Siedler von Catan} als Computerspiel umgesetzt werden und durch KIs spielbar sein. Dabei werden an die KIs die gleichen Ma�st�be angelegt, die auch f�r menschliche Spieler gelten. Ferner wird ihnen das \quotes{Cheaten} nicht erlaubt.

\section{Ziel der Arbeit}
Ziel dieser Bachelorarbeit ist es, ein rundenbasiertes Strategiespiel f�r KI spielbar umzusetzen sowie die Konzeption, Implementation und Evaluation zu dokumentieren. Konkret wurde dabei das bekannte Gemeinschaftsspiel \textit{Siedler von Catan} als Ziel der Implementation gew�hlt. 

In dieser schriftlichen Ausarbeitung werden demnach alle theoretischen Aspekte der Software betrachtet, die dazu umgesetzt wurden. Es soll von den Grundlagen, die zum Verst�ndnis sp�terer Ausf�hrungen wichtig sind, bis zur Evaluation der entworfenen Software alles abgedeckt werden. 

\section{Aufbau der Arbeit}
Zun�chst werden in Kapitel 2 die n�tigen Grundlagen thematisiert, die zum Nachvollziehen sp�terer Kapitel n�tig sind. In Kapitel 3 wird dann auf die Konzeption der Software eingegangen. Hier wird wiederum zwischen einer Problemstellung, welche die Gr�nde und Ziele der Arbeit konkretisiert, und einem Entwurf f�r die Software unterschieden. 

Die Erl�uterung der tats�chlichen Implementation erfolgt anschlie�end im 4. Kapitel, welches im Detail auf alle wichtigen Bestandteile der angefertigten Software eingeht. In diesem Kapitel soll auch ein Vergleich zwischen dem Entwurf und der tats�chlichen Umsetzung erfolgen. 

Im 5. Kapitel soll dann schlie�lich zur Evaluation �bergegangen werden. Diese soll pr�fen, ob die gestellten Anforderungen erf�llt werden konnten und eine Bewertung des Implementierten vornehmen. Die Evaluation wird auf Grundlage der erf�llten bzw. nicht erf�llten Anforderungen und durchgef�hrten Tests vorgenommen. Nach der Bewertung werden Probleme der Entwicklung angebracht, sowie deren Ursachen zu beschrieben.

Nach der Evaluation folgt der Schlussteil in dem die Kernpunkte dieser Ausarbeitung zusammengefasst werden und ein Fazit gezogen wird. Au�erdem wird ein Ausblick gegeben, indem L�sungen zu den in Kapitel 5 beschriebenen Problemen pr�sentiert werden. Au�erdem werden weitere Ans�tze f�r Verbesserungen der Software untersucht. 




