% Erzeugt das Deckblatt
%   Bei zu langem Arbeitstitel m�ssen die vertikalen Abst�nde (\vspace)
%   angepasst werden, damit das Deckblatt weiterhin auf eine Seite passt.
\begin{titlepage}
\newgeometry{top=2cm,bottom=2cm,left=2cm,right=2cm}
\begin{flushleft}
	%\includegraphics[width=6.75cm]{Abbildungen/luh_logo.ps}\\
	{\Large Universit"at Hildesheim \\
		\large Institut f"ur Informatik} \\
	\begin{flushright}\vspace*{-5cm}
		\includegraphics[scale=0.087]{uni_hi_logo.pdf}
	\end{flushright}
\end{flushleft}

\begin{center}
	\vspace{3cm}
	\Huge{\textbf{\titel}}
\end{center}
\begin{center}
    \vspace*{3cm}
    \large{\textbf{\art~im Studiengang \studiengang}}\\
    Sommersemester 2020
    \vspace{1cm}
\end{center}
\begin{center}
vorgelegt von \\[1.5ex] 
\textbf{Tjark Harjes} \\[1.5ex]
Matr.-Nr.: 301249 \\
4. Semester IMIT \\
E-Mail: harjes@uni-hildesheim.de\\[1.5cm]
Hildesheim, den \textbf{07.06.2020} \\ 
\end{center} 
\begin{flushleft}
	\vspace{4cm}
	\textbf{Erstgutachter: } \erstgutachter \\
	\textbf{Zweitgutachter: } \zweitgutachter
\end{flushleft}
\end{titlepage}
\restoregeometry